% This is LLNCS.DEM the demonstration file of
% the LaTeX macro package from Springer-Verlag
% for Lecture Notes in Computer Science,
% version 2.4 for LaTeX2e as of 16. April 2010
%
\documentclass{llncs}

% allows for temporary adjustment of side margins
\usepackage{chngpage}

% just makes the table prettier (see \toprule, \bottomrule, etc. commands below)
\usepackage{booktabs}

\usepackage[utf8]{inputenc}
%\usepackage[font=small,skip=0pt]{caption}

% footnotes
\usepackage{scrextend}

% colors
\usepackage[usenames, dvipsnames]{color}

% underline
\usepackage{tikz}
\newcommand{\udensdot}[1]{%
    \tikz[baseline=(todotted.base)]{
        \node[inner sep=1pt,outer sep=0pt] (todotted) {#1};
        \draw[densely dotted] (todotted.south west) -- (todotted.south east);
    }%
}%

\newcommand{\uloosdot}[1]{%
    \tikz[baseline=(todotted.base)]{
        \node[inner sep=1pt,outer sep=0pt] (todotted) {#1};
        \draw[loosely dotted] (todotted.south west) -- (todotted.south east);
    }%
}%

\newcommand{\udash}[1]{%
    \tikz[baseline=(todotted.base)]{
        \node[inner sep=1pt,outer sep=0pt] (todotted) {#1};
        \draw[dashed] (todotted.south west) -- (todotted.south east);
    }%
}%

\newcommand{\udensdash}[1]{%
    \tikz[baseline=(todotted.base)]{
        \node[inner sep=1pt,outer sep=0pt] (todotted) {#1};
        \draw[densely dashed] (todotted.south west) -- (todotted.south east);
    }%
}%

\newcommand{\uloosdash}[1]{%
    \tikz[baseline=(todotted.base)]{
        \node[inner sep=1pt,outer sep=0pt] (todotted) {#1};
        \draw[loosely dashed] (todotted.south west) -- (todotted.south east);
    }%
}%

% URL handling
\usepackage{url}
\urlstyle{same}

%\usepackage{makeidx}  % allows for indexgeneration

%\usepackage{amsmath}
\usepackage{amsmath, amssymb}
\usepackage{mathabx}
\usepackage{caption} 
\captionsetup[table]{skip=10pt}

% monospace within text
\newcommand{\ms}[1]{\texttt{#1}}

% examples
\usepackage{fancyvrb}
\DefineVerbatimEnvironment{ex}{Verbatim}{numbers=left,numbersep=2mm,frame=single,fontsize=\scriptsize}

\usepackage{xspace}
% Einfache und doppelte Anfuehrungszeichen
\newcommand{\qs}{``} 
\newcommand{\qe}{''\xspace} 
\newcommand{\sqs}{`} 
\newcommand{\sqe}{'\xspace} 

% checkmark
\usepackage{tikz}
\def\checkmark{\tikz\fill[scale=0.4](0,.35) -- (.25,0) -- (1,.7) -- (.25,.15) -- cycle;} 

% Xs
\usepackage{pifont}

% Tabellenabstände kleiner
\setlength{\intextsep}{10pt} % Vertical space above & below [h] floats
\setlength{\textfloatsep}{10pt} % Vertical space below (above) [t] ([b]) floats
% \setlength{\abovecaptionskip}{0pt}
% \setlength{\belowcaptionskip}{0pt}

\usepackage{tabularx}
\newcommand{\hr}{\hline\noalign{\smallskip}} % für die horizontalen linien in tabellen

% Todos
\usepackage[colorinlistoftodos]{todonotes}
\newcommand{\ke}[1]{\todo[size=\small, color=orange!40]{\textbf{Kai:} #1}}
\newcommand{\tb}[1]{\todo[size=\small, color=green!40]{\textbf{Thomas:} #1}}
\newcommand{\er}[1]{\todo[size=\small, color=red!40]{\textbf{Erman:} #1}}
\newcommand{\an}[1]{\todo[size=\small, color=blue!40]{\textbf{Andy:} #1}}

\newenvironment{table-1cols}{
  \scriptsize
  \sffamily
  \vspace{0.3cm}
  \begin{tabular}{l}
  \hline
  \textbf{Requirements} \\
  \hline

}{
  \hline
  \end{tabular}
  \linebreak
}

\newenvironment{table-2cols}{
  \scriptsize
  \sffamily
  \vspace{0.3cm}
  \begin{tabular}{l|l}
  \hline
  \textbf{Requirements} & \textbf{Covering DSCLs} \\
  \hline

}{
  \hline
  \end{tabular}
  \linebreak
}

\newenvironment{complexity}{
  %\scriptsize
  %\sffamily
  %\vspace{0.3cm}
  \begin{tabular}{l|l}
  \hline
  \textbf{Complexity Class} & \textbf{Complexity} \\
  \hline

}{
  \hline
  \end{tabular}
  \linebreak
}

\newenvironment{DL}{
  %\scriptsize
  %\sffamily
  \vspace{0cm}
  \begin{tabular}{r l}

}{
  \end{tabular}
  %\linebreak
}


\newenvironment{evaluation}{
  %\scriptsize
  %\sffamily
  %\vspace{0.3cm}
  \begin{tabular}{l|c|c|c|c|c|c}
  \hline
  \textbf{Constraint Class} & \textbf{DSP} & \textbf{OWL2-DL} & \textbf{OWL2-QL} & \textbf{ReSh} & \textbf{ShEx} & \textbf{SPIN} \\
  \hline

}{
  \hline
  \end{tabular}
  \linebreak
}

\newenvironment{constraint-languages-complexity}{
  %\scriptsize
  %\sffamily
  %\vspace{0.3cm}
  \begin{tabular}{l|c|c|c|c|c|c}
  \hline
  \textbf{Complexity Class} & \textbf{DSP} & \textbf{OWL2-DL} & \textbf{OWL2-QL} & \textbf{ReSh} & \textbf{ShEx} & \textbf{SPIN} \\
  \hline

}{
  \hline
  \end{tabular}
  \linebreak
}

\newenvironment{user-fiendliness}{
  %\scriptsize
  %\sffamily
  %\vspace{0.3cm}
  \begin{tabular}{l|c|c|c|c|c}
  \hline
  \textbf{criterion} & \textbf{DSP} & \textbf{OWL2} & \textbf{ReSh} & \textbf{ShEx} & \textbf{SPIN} \\
  \hline

}{
  \hline
  \end{tabular}
  \linebreak
}

\setcounter{secnumdepth}{5}

% tables
\usepackage{array,graphicx}
\usepackage{booktabs}
\usepackage{pifont}
\newcommand*\rot{\rotatebox{90}}
\newcommand*\OK{\ding{51}}
\usepackage{booktabs}
\newcommand*\ON[0]{$\surd$}

\usepackage{tablefootnote}

\usepackage{float}

\begin{document}
\renewcommand{\arraystretch}{1.3}
%
%
\title{Evaluation of Australian Climate Observations Reference Network - Surface Air Temperature Dataset (ACORN-SAT) RDF Data Sets}
\subtitle{}

\titlerunning{XXXXX}  % abbreviated title (for running head)
%                                     also used for the TOC unless
%                                     \toctitle is used
%
\author{Thomas Bosch\inst{1} \and Benjamin Zapilko\inst{1} \and Joachim Wackerow\inst{1} \and Kai Eckert\inst{2}}
%
\authorrunning{} % abbreviated author list (for running head)
%
%%%% list of authors for the TOC (use if author list has to be modified)
\institute{GESIS – Leibniz Institute for the Social Sciences, Germany\\
\email{\{firstname.lastname\}@gesis.org},\\ 
\and
University of Mannheim, Germany \\
\email{kai@informatik.uni-mannheim.de} 
}

\maketitle              % typeset the title of the contribution

\begin{abstract}
Evaluation of Australian Climate Observations Reference Network - Surface Air Temperature Dataset (ACORN-SAT) RDF Data Sets


\keywords{RDF Validation, RDF Constraints, DDI-RDF Discovery Vocabulary, Disco, RDF Data Cube Vocabulary, Linked Data, Semantic Web}
\end{abstract}

\section{Data Model Consistency}

\begin{table}[H]
    \begin{center}
    \begin{tabular}{@{}lccccccccccc@{}}
           & \multicolumn{11}{c}{\textbf{Constraints}}
    \\  \cmidrule{2-12}
    \\       \textbf{Data Sets}
           & \rot{\emph{DATA-MODEL-CONSISTENCY-01}}
           & \rot{\emph{DATA-MODEL-CONSISTENCY-02}}
           & \rot{\emph{DATA-MODEL-CONSISTENCY-03}}
           & \rot{\emph{DATA-MODEL-CONSISTENCY-04}}
           & \rot{\emph{DATA-MODEL-CONSISTENCY-05}}
           & \rot{\emph{DATA-MODEL-CONSISTENCY-06}}
           & \rot{\emph{DATA-MODEL-CONSISTENCY-07}}
           & \rot{\emph{DATA-MODEL-CONSISTENCY-08}}
           & \rot{\emph{DATA-MODEL-CONSISTENCY-09}}
           & \rot{\emph{DATA-MODEL-CONSISTENCY-10 (!)}}
           & \rot{\emph{DATA-MODEL-CONSISTENCY-11}}
	\\ \midrule
    \emph{http://lab.environment.data.gov.au/sparql} & $\checkmark$ & 8 & $\checkmark$ & $\checkmark$ & $\checkmark$ & $\checkmark$ & $\checkmark$ & $\checkmark$ & $\checkmark$ & - & $\checkmark$  \\
    \bottomrule
    \end{tabular}
    \caption{Evaluation of \emph{http://lab.environment.data.gov.au/sparql}}
    \label{tab:evaluation-lab.environment.data.gov.au-sparql}
    \end{center}
\end{table}

\section{Existential Quantifications}

\begin{table}[H]
    \begin{center}
    \begin{tabular}{@{}lcccc@{}}
           & \multicolumn{4}{c}{\textbf{Constraints}}
    \\  \cmidrule{2-5}
    \\       \textbf{Data Sets}
           & \rot{\emph{EXISTENTIAL-QUANTIFICATIONS-01}}
           & \rot{\emph{EXISTENTIAL-QUANTIFICATIONS-02}}
           & \rot{\emph{EXISTENTIAL-QUANTIFICATIONS-03}}
           & \rot{\emph{EXISTENTIAL-QUANTIFICATIONS-04}}
	\\ \midrule
    \emph{http://lab.environment.data.gov.au/sparql} & $\checkmark$ & $\checkmark$ & 4 & $\checkmark$  \\
    \bottomrule
    \end{tabular}
    \caption{Evaluation of \emph{http://lab.environment.data.gov.au/sparql}}
    \label{tab:evaluation-lab.environment.data.gov.au-sparql}
    \end{center}
\end{table}

\section{Cardinality Restrictions}

\begin{table}[H]
    \begin{center}
    \begin{tabular}{@{}lcccc@{}}
           & \multicolumn{4}{c}{\textbf{Constraints}}
    \\  \cmidrule{2-5}
    \\       \textbf{Data Sets}
           & \rot{\emph{MINIMUM-QUALIFIED-CARDINALITY-RESTRICTIONS-02}}
           & \rot{\emph{MAXIMUM-QUALIFIED-CARDINALITY-RESTRICTIONS-01}}
           & \rot{\emph{EXACT-UNQUALIFIED-CARDINALITY-RESTRICTIONS-01}}
           & \rot{\emph{EXACT-QUALIFIED-CARDINALITY-RESTRICTIONS-02}}
	\\ \midrule
    \emph{http://lab.environment.data.gov.au/sparql} & $\checkmark$ & $\checkmark$ & $\checkmark$ & $\checkmark$  \\
    \bottomrule
    \end{tabular}
    \caption{Evaluation of \emph{http://lab.environment.data.gov.au/sparql}}
    \label{tab:evaluation-lab.environment.data.gov.au-sparql}
    \end{center}
\end{table}

\section{Structure}

\begin{table}[H]
    \begin{center}
    \begin{tabular}{@{}lcc@{}}
           & \multicolumn{2}{c}{\textbf{Constraints}}
    \\  \cmidrule{2-3}
    \\       \textbf{Data Sets}
           & \rot{\emph{STRUCTURE-01}}
           & \rot{\emph{STRUCTURE-02}}
	\\ \midrule
    \emph{http://lab.environment.data.gov.au/sparql} & $\checkmark$ & $\checkmark$  \\
    \bottomrule
    \end{tabular}
    \caption{Evaluation of \emph{http://lab.environment.data.gov.au/sparql}}
    \label{tab:evaluation-lab.environment.data.gov.au-sparql}
    \end{center}
\end{table}

\section{Further Constraints}

%\bibliography{../../literature/literature}{}
\bibliographystyle{plain}
\setcounter{tocdepth}{1}
%\listoftodos
\end{document}
